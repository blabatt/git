\section{Share \& Synchronize}
Major sync work-flows:
\begin{itemize}
    \item Centralized -- any can push to a shared bare
    \item Managed -- only manager can push
    \item Hierarchical -- multiple levels of pushing
\end{itemize}


%%%%%%%%%%%%%%%%%%%%%%%%%%%%%%%%%%%%%%%%%%%%%%
\subsection*{git fetch}
\code{git fetch <remote>} \# download latest \\
\code{git fetch <remote> <refspec>} \# specific rqst
\code{git fetch -{}-all} \# all @ once


%%%%%%%%%%%%%%%%%%%%%%%%%%%%%%%%%%%%%%%%%%%%%%
\subsection*{git push}
Only works if: \begin{itemize}[partopsep=0pt,topsep=0pt,parsep=0pt]
    \item you have write access to <to>
    \item no on has pushed since your last fetch
\end{itemize}
If 2\textsuperscript{nd} condition isn\textquotesingle t true, it is an attempted push of \say{non-fast-forward} changes.\\
\code{git push <to> <from>} \# generally \\
\code{git push origin master} \# specifically \\
\code{git push -u origin my\_brn} \# set remote \& push\\
\code{git push <refspec>} \# eg, frombrn:tobrn \\
Push message includes a statement like:\\
\code{  1edee6b..fbff5bc  master -> master}\\
ie, <oldref>..<newref> <frombr> -> <tobr>\\
Pushes on rebased branches require forcing:\\
\code{git push -f myfork brnA} \# \underline{f}orce


%%%%%%%%%%%%%%%%%%%%%%%%%%%%%%%%%%%%%%%%%%%%%%
\subsection*{git pull}
\code{git pull} \# git fetch + git merge \\
\code{git pull <url>} \# one-time: don\textquotesingle t set remote


%%%%%%%%%%%%%%%%%%%%%%%%%%%%%%%%%%%%%%%%%%%%%%
\subsection*{git remote}
\code{git remove -v} \# show all remotes \\
\code{git remote add [<nickname>?] <url>} \# add \\
\code{git remote show <remote>} \# inquire \\
\code{git remote rename <old> <new>} \# new alias \\
\code{git remote rm} \# no longer track remote \\
\code{git remote set-head} \# \\
\code{git remote prune} \# \\
\code{git remote set-branches} \#


%%%%%%%%%%%%%%%%%%%%%%%%%%%%%%%%%%%%%%%%%%%%%%
\subsection*{git request-pull}
Absent GitHub forks \& pull-requests, use this to facilitate the integration process. (A fork is just a clone to or push to an accessible repo.)
\code{git request-pull origin/master myfork} \# text

%%%%%%%%%%%%%%%%%%%%%%%%%%%%%%%%%%%%%%%%%%%%%%
\subsection*{GitHub}
\textit{Generally, pull requests don't close right away. The idea is to open, collaborate, \& finally merge. Use markdown to reference IDs \& prettify:}
{\scriptsize
\begin{itemize}
    \item \code{\#<id>} reference branch, commit, or PR
    \item \code{username\#<refid>} more specifically
    \item \code{username/rep\#<id>} or externally
    \item \code{<SHA1>} unchanging, concretely
    \item \code{:<emo>:} enter an emoji
    \item \code{[X] item A [] item B} or checklist
    \item \code{java```if(true) a=b;```} code snippet
    \item \code{> whan that aprile} block quote
\end{itemize}}

\textit{\say{Trackback} event is then created in referenced conversation.}
% \code{} \# \\

