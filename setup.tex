\section{Setup}


%%%%%%%%%%%%%%%%%%%%%%%%%%%%%%%%%%%%%%%%%%%%%%%%%%%
\subsection*{git config}
Manipulate configuration settings.\\
\code{git config -{}-system <cfg>} \# in /etc/gitconfig \\
\code{git config -{}-global <cfg>} \# in \textasciitilde /.gitconfig \\
\code{git config -{}-local<cfg>} \# in \textasciitilde .git/config \\
\# where <cfg> is <key> <value>, for example: \\
\code{git config -{}-global core.editor vim} \\
\code{git config <key>} \# query value \\
\code{git config -{}-list} \# show all cfgs \\
\code{git config -{}-list -{}-show-origin} \# cfg origin\\
\# Creating aliases:\\
\code{git config -{}-global alias.<als> <cmd>} \\
\code{git config alias.l log -{}-oneline} \# examples \\
\code{git config -{}-global alias.unstage \textquotesingle reset HEAD -{}- \textquotesingle} \\
\# Simplify long commands: \\
\code{git config -{}-local remote.pushDefault origin} \\
Run \code{man git-config} for more config options.

%%%%%%%%%%%%%%%%%%%%%%%%%%%%%%%%%%%%%%%%%%%%%%%%%%%
\subsection*{git help}
\code{git help <cmd>} \# general inquiry \\
\code{git help config} \# eg: help on \say{config} \\
\code{git <cmd> -h} \# abridged help 


%%%%%%%%%%%%%%%%%%%%%%%%%%%%%%%%%%%%%%%%%%%%%%%%%%%
\subsection*{git init}
Start new git repository locally. \\
\$ \code{mkdir ~/Projects/moustrap} \# create \\
\$ \code{cd ~/Projects/moustrap} \# relocate \\
\$ \code{vim innovation.rs} \# innovate \\
\code{git init} \# track \\
Create a central hub repo, ie one you can push to that lacks a working dir: \\
\$ \code{mkdir ~/Projects/central} \# create \\
\$ \code{cd ~/Projects/central} \# relocate \\
\code{git init -{}-bare} \# no working dir.


%%%%%%%%%%%%%%%%%%%%%%%%%%%%%%%%%%%%%%%%%%%%%%%%%%%
\subsection*{git clone}
Grab \& track an existing, remote git repository. \\
\code{git clone <url>} \# store locally in \textbackslash \\
\code{git clone <url> <name>} \# set local <name> \\
\code{git clone <url> -o <name>} \# set remote name\\
\code{git clone -{}-bare <url>} \# no working dir \\
\code{git clone ssh://[user@]server/project.git} \# SSH


%%%%%%%%%%%%%%%%%%%%%%%%%%%%%%%%%%%%%%%%%%%%%%%%%%%
\subsection*{.gitignore}
Pattern matching rules:
\begin{itemize}
    \item comment out line with \code{\#}
    \item standard globs  (\code{*}, \code{?}) work recursively
    \item avoid recursion by prefacing with \code{\textbackslash}
    \item match directory recursion by \code{**}
    \item specify directory by suffixing with \code{\textbackslash}
    \item negate a pattern with \code{!}
    \item case matching with \code{[abc]} or \code{[a-c]}
\end{itemize}
For example: \\
(file) \code{*.a} \# ignore all *.a \\
(file) \code{!lib.a} \# do track lib.a \\
(file) \code{/TODO} \# ignore top-level /TODO\\
(file) \code{build/} \# ignore anything under build/ \\
(file) \code{doc/*.txt} \# ignore doc txt's \\
(file) \code{doc/**/*.pdf} \# ignore all pdf's


%%%%%%%%%%%%%%%%%%%%%%%%%%%%%%%%%%%%%%%%%%%%%%%%%%%
\subsection*{Key Management}
\code{ssh-keygen -o} \# create pub/prv key pair\\
\code{cat \textasciitilde/.ssh/id\_rsa.pub} \# view pub key


%%%%%%%%%%%%%%%%%%%%%%%%%%%%%%%%%%%%%%%%%%%%%%%%%%%
\subsection*{Server Administration}
There are several ways to serve git (and cons):
\begin{enumerate}
    \item The local filesystem (no redundancy)
    \item An SSH server (cumbersome user admin)
    \item GIT protocol (no user admin possible)
    \item \say{Smart} HTTP (complex web server)
    \item Self-hosting, eg GitLab (time to set up)
    \item Commercial hosting, eg Github (money)
\end{enumerate}

\# Set up remote access SSH server:\\
\code{mkdir .ssh \&\& chmod 700 .ssh \&\& /}\\
\code{touch .ssh/authorized\_keys \&\& /}\\
\code{chmod 600 /ssh/authorized\_keys}\\
\# Allow jack access to ssh user:\\
\code{cat /tmp/id\_rsa.jack.pub >> /} \\
\code{\textasciitilde/.ssh/authorized\_keys} \\
\# Set up Git-protocol server:\\
\code{git daemon -{}-reuseaddr <path>}\\
\code{cd prj\_path/ \&\& touch git-daemon-export-ok}
