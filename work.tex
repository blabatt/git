\section{Work}
Terminology \& state machine: \\
{\scriptsize
\begin{tabular}{l l l l}
    State & Index & W-Dir & Next \\ 
    \hline
    Untracked & ? & ? & Add \\
    Unmodified & - & - & Edit \\
    Modified & - & M & Stage \\
    Staged & [A|M] & [M]? & Commit \\
\end{tabular}
}\\
You can \code{rm} from \say{untracked} or \code{git rm} from any other state.

%%%%%%%%%%%%%%%%%%%%%%%%%%%%%%%%%%%%%%%%%%%%%
\subsection*{git reset}
\textit{Reset moves HEAD's branch \& HEAD, \href{https://stackoverflow.com/questions/3639342/whats-the-difference-between-git-reset-and-git-checkout}{versus checkout}, which only moves HEAD.  Below, <ref> $\in$ \{SHA1, branch, file path\}:}\\
\code{git reset <ref>} \# move HEAD bch \& idx to ref\\ 
\code{git reset -{}-mixed <ref>} \# equiv to above\\
\code{git reset -{}-soft <ref>} \# Move HEAD brch $\neg$ idx\\
\code{git reset -{}-hard} \# revert WD, idx \& brch \\[2mm]
\textit{Only the -{}-hard option is dangerous to WD. Revert specific file in working directory:}\\
\code{git reset <file>} \# file in WD $\Delta$s to last HEAD\\
\code{git reset <SHA> <file>} \# revert to specific SHA \\
\code{git checkout -{}- <file>} \# \say{unsafely} revert file \\

%%%%%%%%%%%%%%%%%%%%%%%%%%%%%%%%%%%%%%%%%%%%%
\subsection*{\href{https://git-scm.com/docs/git-revert\#_examples}{git revert}}

\textit{Append a \href{https://stackoverflow.com/questions/19032296/how-to-use-git-revert}{new commit} to history with contents of an older one. \code{git revert} is \href{https://stackoverflow.com/questions/8358035/whats-the-difference-between-git-revert-checkout-and-reset}{unrelated} to \code{restore} or \code{reset}. To get \ul{rid} of \code{HEAD}'s content:} \\
\code{git revert HEAD}
% \code{} \# \\
% \code{} \# \\
% \code{} \# \\

%%%%%%%%%%%%%%%%%%%%%%%%%%%%%%%%%%%%%%%%%%%%%
\subsection*{git restore}
\textit{\href{https://git-scm.com/book/en/v2/Git-Basics-Undoing-Things}{Safe way} to restore old file version:}\\
\code{git restore CONTRIBUTING.md} \# \textit{from HEAD} \\
\code{git restore -{}-staged CONTRIBUTING.md} \# \textit{from idx}
% \code{} \# \\
% \code{} \# \\
% \code{} \# \\

%%%%%%%%%%%%%%%%%%%%%%%%%%%%%%%%%%%%%%%%%%%%%
\subsection*{git add}
\code{git add <filename>} \# add to index\\
\code{git add .} \# stage all changed\\
\code{git add -{}-all} \# stage all\\
\code{git add -i} \# interactive staging

%%%%%%%%%%%%%%%%%%%%%%%%%%%%%%%%%%%%%%%%%%%%%
\subsection*{git commit}
\code{git commit -m 'message'} \# commit all staged \\
\code{git commit -am 'message'} \# commit all tracked \\
\code{git commit -a -{}-amend} \# redo last commit \\
\code{git commit -{}-amend} \# tweak last commit 

%%%%%%%%%%%%%%%%%%%%%%%%%%%%%%%%%%%%%%%%%%%%%%%
\subsection*{git tag}
\code{git tag} \# show tags \\
\code{git tag -l} \# same as above\\
\code{git tag -l <tag\_ref>} \# show all matches \\
\code{git tag \textquotedbl <str>\textquotedbl} \# create \say{lightweight} tag \\
\code{git tag -a <str> -m \textquotedbl <str>\textquotedbl} \# \say{annotated} tag \\
\code{git tag -a <tag> <hash>} \# after-the-fact tag \\
\# using tags:\\
\code{git show <tag>} \# display commit @ <tag>\\
\code{git checkout <tag>} \# checkout tagged commit \\
\# remote tags:\\
\code{git push origin <tag>} \# explicit push required \\
\code{git push <remote> -{}-tags} \# push all tags \\
\# deleting tags: \\
\code{git tag -d <tag>} \# delete locally \\
\code{git push <remote> -{}-delete <tag>} \# remotely \\
\# sign tags (must disseminate pub key): \\
\code{git tag -s v1.5} \# sign

%%%%%%%%%%%%%%%%%%%%%%%%%%%%%%%%%%%%%%%%%%%%%%%
\subsection*{git rm}
\code{git rm <file>} \# stage <file>'s removal \\
\code{git rm -f <file>} \# remove modified / staged \\
\code{git rm -{}-cached <file>} \# untrack \\
Can use globs instead of filenames:\\
\code{git rm \textbackslash*~} \# anything ending in \textasciitilde

%%%%%%%%%%%%%%%%%%%%%%%%%%%%%%%%%%%%%%%%%%%%%%%
\subsection*{git mv}
\code{git mv <old> <new>} \# rename to <new>\\
This is shorthand for:\\
\$ \code{git rm \_; mv \_ \_; git add \_} \\
on a tracked file.


%%%%%%%%%%%%%%%%%%%%%%%%%%%%%%%%%%%%%%%%%%%%%%%
\subsection*{git stash}
\code{git stash push} \# \\
\code{git stash list} \# \\
\code{git stash apply} \# \\
\code{git stash drop} \# remove from stack \\
\code{git stash pop} \# apply \& drop last

%%%%%%%%%%%%%%%%%%%%%%%%%%%%%%%%%%%%%%%%%%%%%%%
\subsection*{git cherry-pick}
Like a rebase for a single commit. Can cherry-pick multiple commits into one prior to integration, or cherry-pick single commit onto master instead of rebasing. \\
\code{git cherry-pick <SHA1>} \#

% \code{} \# \\
