\section{Branch}
Types of commits:
\begin{itemize}
    \item initial commit ... no parent
    \item normal commit ... one parent
    \item merge commit ... multiple parents
\end{itemize}
Types of branches:
\begin{itemize}
    \item \say{Long-running} branches
    \item \say{Topic} branches
    \item \say{Tracking} branches
\end{itemize}
Types of history:
\begin{itemize}
    \item Original, unedited (3-way merges)
    \item Curated, linearized (rebase, cherry-pick)
\end{itemize}

%%%%%%%%%%%%%%%%%%%%%%%%%%%%%%%%%%%%%%%%%%%%%
\subsection*{git branch}
\code{git branch -v} \# show branches \\
\code{git branch -vv} \# show tracking branches \\
\code{git branch -{}-merged} \# end-of-road branches \\
\code{git branch -{}-no-merged} \# show WIP branches \\
\code{git branch <bname>} \# new branch @ HEAD \\
\code{git branch -d <branch>} \# delete \\
\# Set upstream branch:\\
\code{git branch -u <remote>/<branch>}


%%%%%%%%%%%%%%%%%%%%%%%%%%%%%%%%%%%%%%%%%%%%%
\subsection*{git checkout}
\textit{Moves HEAD to a named branch or SHA commit.  You cannot checkout a branch that clashes with changes in your working directory.}\\
\code{git checkout <branch>} \# move HEAD \\
\code{git checkout -b <bname>} \# create \& checkout \\[2mm]
\textit{To \href{https://stackoverflow.com/questions/215718/how-can-i-reset-or-revert-a-file-to-a-specific-revision}{restore older version} of just one file:}\\
\code{git checkout [commit ID] -{}- path/to/file}\\
\code{git \href{https://git-scm.com/docs/git-restore\#_examples}{restore} path/to/file \# ibid, \href{https://stackoverflow.com/questions/48508799/how-to-reset-all-files-from-working-directory-but-not-from-staging-area/57066072\#57066072}{new in v2.23}}\\[2mm]
\# Create remote tracking branch:\\
\code{git checkout -b <name> <remote>/<branch>} \\
\code{git switch -c <bname>} \# \href{https://stackoverflow.com/questions/3965676/why-did-my-git-repo-enter-a-detached-head-state}{ibid}\\
\code{git checkout -{}-track <remote>/<branch>}



%%%%%%%%%%%%%%%%%%%%%%%%%%%%%%%%%%%%%%%%%%%%%
\subsection*{git merge}
\code{git merge <branch> [<into>]} \# * assumed \\
A merge is three-way, generally, but a two-way \say{fast-forward} when <branch> is a direct ancestor of <into>.\\ 
\code{git merge @\{u\}} \# shorthand for upstream \\
\code{git mergetool} \# visual conflict-resolution \\
\code{git merge -{}-squash brnA} \# single parent



%%%%%%%%%%%%%%%%%%%%%%%%%%%%%%%%%%%%%%%%%%%%%
\subsection*{git rebase}
{\footnotesize 
Rebasing replays changes from one line of work onto another in the order they were introduced.  
Only rebase if it\textquotesingle s the project's convention to do so and if you want to clean up your clutter. \textbf{Never rebase anything you've pushed} (although it's technically still okay if no one else will base their work on those commits).}\\
\# Rebase <topic> onto <base>:\\ 
\code{git rebase <base> <topic>} \# general form \\
\# Now, <base> is fast-forwardable:\\
\code{git checkout <base>; git merge <topic>}\\
\code{git rebase master} \# rebase * onto master\\
\# Rebase c onto a, holding off on b: \\
\code{git rebase -{}-onto <a> <b> <c>} \\
\# Split, squash, edit, reorder commits: \\
\code{git rebase -i ...} \# \underline{i}nteractive


%%%%%%%%%%%%%%%%%%%%%%%%%%%%%%%%%%%%%%%%%%%%%
\subsection*{git merge-base}
Finds common ancestor:\\
\code{git merge-base brn master} \# find \\
\code{git diff \$\{git merge-base brn master\}} \# use \\
\code{git diff master...brn} \# shorthand for above 

%\code{} \# \\
