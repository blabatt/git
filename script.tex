\section{Scripting}

%%%%%%%%%%%%%%%%%%%%%%%%%%%%%%%%%%%%%%%
\subsection*{Hooks}
{\footnotesize 
\textbf{Client Side}: Pre and post- events like commit, push, etc, git can run script in .git/hooks/ directory.  These are excluded from version control. \textbf{Server Side}: Similar to client side, but are included in pushes, and used to implement controlled team processes.
\textbf{GitHub}: Upon <event> (often a push), GitHub emits an HTTP post to indicated url.  The server listening there can parse, interpret \& act according to arbitrary logic. 
}

%%%%%%%%%%%%%%%%%%%%%%%%%%%%%%%%%%%%%%%
\subsection*{Github API}
GET a public HTTP \underline{e}nd \underline{p}oint like this:\\
\code{git curl <EP>} \# basic api GET request \\
Or authenticate (token or OAuth) then POST: \\
\code{curl -H \textquotedbl ContentType: application/json \textquotedbl  \textbackslash }\\
\phantom{xxxx} \code{-H \textquotedbl Authorization: token TOK \textquotedbl  \textbackslash}\\
\phantom{xxxx} \code{-{}-data '<json>' \ <EP>} 

Top level end-points:
\begin{multicols}{2}
\scriptsize
\begin{itemize}
    \item activity
    \item data
    \item gists
    \item actions
    \item interactions
    \item issues
    \item organizations
    \item projects
    \item PRs
    \item reactions
    \item teams
    \item users
\end{itemize}
\end{multicols}
\ \\